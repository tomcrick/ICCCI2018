\documentclass{llncs}

\usepackage{graphicx}
\usepackage[hyphens]{url}
\usepackage{booktabs}
\usepackage{paralist}

% natbib for refs
%\usepackage[numbers,sort]{natbib} 

\begin{document}

\title{{\emph{Immigration Immigration Immigration!}}: Popularity and
  Geospatial Spread of Trends on Twitter}

\author{Nabeel Albishry\inst{1,3}\thanks{This work has been supported by a doctoral research scholarship for
Nabeel Albishry from King Abdulaziz University, Kingdom of Saudi
Arabia.} \and Tom
  Crick\inst{2} \and Teleem Fagade\inst{1} \and Theo Tryfonas\inst{1}}


\institute{Faculty of Engineering, University of Bristol, UK\\\email{\{n.albishry,tesleem.fagade,theo.tryfonas\}@bristol.ac.uk}
\and 
Department of Computer Science, Swansea University, UK\\\email{thomas.crick@swansea.ac.uk}
\and
Faculty of Computing \& IT, King Abdulaziz University, Jeddah, Saudi Arabia\\\email{nalbishry@kau.edu.sa}}
\maketitle

\begin{abstract}
The techniques presented here helps giving broad understanding of
public concerns and common issue across different places. Hundreds of
topics trend throughout the day, which makes it nearly impossible to
analysis eve-rything out there. Hence, the necessity of more generic
approach that is quick and less costly were main motivation behind the
work presented here. Therefore, all analysis and methods explained in
this paper are based on trends information only and do not include
processing of individual topics. With a year-worth of trends data,
this work investigates, popularity and geospatial spread of
trends. The findings show that spread likelihood of trend to other
places is, to some extent, influenced by the place at with it first
appeared. To the best of our knowledge, this is the first study to
ad-dress geospatial spread of trends on Twitter.
 \end{abstract}

\begin{keywords}
Trends, topic spread, network graphs
\end{keywords}

\section{Introduction}\label{intro}

With the tremendous volume of tweets, trending topics serve as
valuable sources of information on summarising what is going on
Twitter, worldwide or in specific loca-tion. Apart from official trend
lists provided by the platform, on the website or through API
endpoints, trends and topics detection have been receiving a lot of
atten-tion across different research domains. In the health field for
example, monitoring and predicting of trending topics through social
media are adopted to measure health is-sues, such as flu spread, and
therefore have gained considerable attention in the recent years
[1]–[4]. Also, in marketing and business domain, topic detection and
classifica-tion are important approaches to extract knowledge about
public opinions from posts on social media [5], as well as in
analysing voting intentions and political view of users [6].

With the increasing popularity of social networks, effect of trends on
public have placed them in many social media campaigns and public
relations strategy. This has made trends a valuable target for
manipulation [7] stuffing [8], spamming [9][10] and hijacking
[11]. For example, the study in [12] explored a trend hijacking case
and suggested that increasing social media engagement may not be
beneficial for public relations.

A common approach in analysing Twitter trends is by clustering and
classification of trending topics based on content [13][14][15]. Also,
the study in [16] presented a content-independent method to model
trends progression through dynamics of users interactions. Other
studies aimed to provide real time classification or detection of
trends, such as [17][18]. With the increasing need for trends analysis
across various domains, customisable clustering tools that can be used
by non-technical users started to emerge, such as the recent example
introduced in [19].


% \begin{acks}
% This work has been supported by a doctoral research scholarship for
% Nabeel Albishry from King Abdulaziz University, Kingdom of Saudi
% Arabia.
% \end{acks}


% bib
\bibliographystyle{splncs}
\bibliography{iccci2018}

\end{document}
